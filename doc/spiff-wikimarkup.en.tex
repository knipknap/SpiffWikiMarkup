\input{spiff-wikimarkup.tex}   % Import common styles.
\fancyfoot[C]{Page \thepage}
\title{\productname\ Release \productversion\\
User Documentation\\
\vspace{5 mm}
\large A Python module for converting between WikiMarkup and HTML}
\author{Samuel Abels}

\begin{document}
\maketitle
\tableofcontents

\newpage
\section{Introduction}
\subsection{Why \productname?}

\product provides is a parser that bidirectionally converts between HTML and
a wiki markup language.


\subsection{Legal Information}

\product and this handbook are distributed under the terms and conditions 
of the GNU GPL (General Public License) Version 2. You should have received 
a copy of the GPL along with \product. If you did not, you may read it here:

\vspace{1em}
\url{http://www.gnu.org/licenses/gpl-2.0.txt}
\vspace{1em}

If this license does not meet your requirements you may contact us under 
the points of contact listed in the following section. Please let us know 
why you need a different license - perhaps we may work out a solution 
that works for either of us.


\subsection{Contact Information \& Feedback}

If you spot any errors, or have ideas for improving \product or this 
documentation, your suggestions are gladly accepted.
We offer the following contact options: \\

\input{contact.tex}

\newpage
\section{Overview}

\begin{lstlisting}
from SpiffWikiMarkup import Html2Wiki, Wiki2Html

parser = Html2Wiki()
parser.feed('''<b>test</b>\n''')
self.assert_(parser.wiki == '''*test*''')

parser = Wiki2Html()
parser.read('test.wiki')
print parser.html
\end{lstlisting}

\end{document}
